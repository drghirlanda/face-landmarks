% Created 2017-02-21 Tue 21:04
\documentclass[11pt]{article}
\usepackage[utf8]{inputenc}
\usepackage[T1]{fontenc}
\usepackage{fixltx2e}
\usepackage{graphicx}
\usepackage{longtable}
\usepackage{float}
\usepackage{wrapfig}
\usepackage{rotating}
\usepackage[normalem]{ulem}
\usepackage{amsmath}
\usepackage{textcomp}
\usepackage{marvosym}
\usepackage{wasysym}
\usepackage{amssymb}
\usepackage{hyperref}
\usepackage{mathptmx}
\tolerance=1000
\usepackage{color}
\usepackage{listings}
\usepackage[margin=1in]{geometry}
\author{Stefano Ghirlanda}
\date{\today}
\title{Project Comments}
\hypersetup{
  pdfkeywords={},
  pdfsubject={},
  pdfcreator={Emacs 24.5.1 (Org mode 8.2.10)}}
\begin{document}

\maketitle
\setlength{\parindent}{0pt}
\setlength{\parskip}{2ex}

I see that you have switched to using \verb~data.table~ and the \verb~apply~
function to aggregate data. That's great, but \verb~data.table~ has many
more features that can simplify our work greatly. I'll explain
below. 

Using \verb~data.table~ can help us fix some issue with your current code:

\begin{enumerate}
\item We will not have always four files for each face. So we need an
approach that does not hard-code the number of files for each
face. This is very easy using \verb~data.table~.

\item Using \verb~data.table~, the calculations of \verb~sd~ and other quantities
can be streamlined even more. \verb~data.table~ is also coded in C, very
efficiently, and it is \emph{very} fast.

\item Landmark files will normally not be in the current directory. We
should have a variable such as \verb~data.dir~ and then use 
\verb~paste0( data.dir, "/*.xlsx" )~ to generate the string that is
passed to \verb~list.files~. (This point is not related to \verb~data.table~.)
\end{enumerate}

Below I will guide you to solve these problems. I am not going to
write the code for you :) but I will tell you how to use \verb~data.table~,
and I will provide some examples and pseudo-code. The idea is that,
after studying a bit what I write below, you should be able to write
more concise and more generalizable code using \verb~data.table~. You goal
is to solve points 1--3 above. Please let me know if you have any
questions!

(As an aside: you don't need to generate a separate results file for
each face, but we can talk about this later.)

\section*{A note on separation of function}
\label{sec-1}

In your current code, you read and process one file at a time, but I
think it is cleaner to split the code into two separate stages, data
acquisition and data processing. Pseudo-code for data acquisition can
be as follows:

\lstset{language=R,label= ,caption= ,numbers=none}
\begin{lstlisting}
landmarks <- NULL # data structure, initially empty
for( file in file.list ) {
    ## 1. read file
    ## 2. append file to landmarks
}
\end{lstlisting}

For step 1, just use \verb~read_excel~ as you are doing already. Say that
the result is in variable \verb~x~. For step 2, use the function \verb~rbind~
(bind by row) to glue together \verb~x~ and \verb~landmarks~:

\lstset{language=R,label= ,caption= ,numbers=none}
\begin{lstlisting}
## step 2. in the loop above:
landmarks <- rbind( landmarks, x )
\end{lstlisting}

The loop then produces a \verb~landmarks~ data structure that we can
convert to \verb~data.table~ for further processing:

\lstset{language=R,label= ,caption= ,numbers=none}
\begin{lstlisting}
landmarks <- data.table( landmarks )
\end{lstlisting}

At this point you may ask: but now how do I know which face a data
point refers to? The answer is that the landmarks file have been
designed with this kind of data processing in mind. Each line in a
landmark file includes information about which point and which face it
belongs to, and about who marked the point. Using this information,
\verb~data.table~ can do quite a bit of magic for us. 

\section*{Data table summary}
\label{sec-2}

\subsection*{Introduction}
\label{sec-2-1}

\begin{itemize}
\item A regular \verb~data.frame~ is naturally indexed as a 2D array: if \verb~x~ is
a \verb~data.frame~, then \verb~x[i,j]~ is the elemnt in row $i$ and column
$j$.
\item In other words, the \verb~data.frame~ operator \verb~[]~ is used for indexing.
\item The \verb~data.table~ operator \verb~[]~, on the other hand, is redefined to
also enable computing on the contents of the \verb~data.table~.
\item This operator can take 3 arguments, rather than 2, that are usually
called (from first to last):
\begin{itemize}
\item the \verb~i~ argument
\item the \verb~j~ argument
\item the \verb~by~ argument
\end{itemize}
\end{itemize}

We can explain these arguments with a few examples. In the following,
you will see R code and, right below, the results it
produces. Consider the following data assumed to be in a \verb~data.frame~
called \verb~d~:

\begin{center}
\begin{tabular}{llr}
Person & Sex & Age\\
\hline
John & M & 20\\
Jack & M & 22\\
Ann & F & 21\\
Sue & F & 30\\
\end{tabular}
\end{center}

We first convert to \verb~data.table~:

\lstset{language=R,label= ,caption= ,numbers=none}
\begin{lstlisting}
library(data.table)
d <- data.table( data )
\end{lstlisting}

\subsection*{Use the first argument to index:}
\label{sec-2-2}

You can index using the \verb~i~ argument. For example, you can index by
\verb~Sex~:
\lstset{language=R,label= ,caption= ,numbers=none}
\begin{lstlisting}
d[ Sex=="F" ]
\end{lstlisting}

\begin{verbatim}
   Person Sex Age
1:    Ann   F  21
2:    Sue   F  30
\end{verbatim}

Or you can index by age:
\lstset{language=R,label= ,caption= ,numbers=none}
\begin{lstlisting}
d[ Age>21 ]
\end{lstlisting}

\begin{verbatim}
   Person Sex Age
1:   Jack   M  22
2:    Sue   F  30
\end{verbatim}

Note that the variables \verb~Sex~ and \verb~Age~ are automatically in scope (if
\verb~d~ had been a simple \verb~data.frame~, you would have had to write 
\verb~d[ d$Sex=="F" ]~ etc.).

\subsection*{Use the second argument to perform computations:}
\label{sec-2-3}

You can perform computations on the data using the \verb~j~ argument:
\lstset{language=R,label= ,caption= ,numbers=none}
\begin{lstlisting}
d[ Sex=="M", mean(Age) ]
\end{lstlisting}

\begin{verbatim}
[1] 21
\end{verbatim}

And you can even perform multiple computations, using the syntax \verb~.()~
to build lists of results:
\lstset{language=R,label= ,caption= ,numbers=none}
\begin{lstlisting}
d[ Sex=="M", .( mean(Age), sd(Age) ) ]
\end{lstlisting}

\begin{verbatim}
   V1       V2
1: 21 1.414214
\end{verbatim}

The result is a \verb~data.table~, in which the results of your computation
have been assigned names \verb~V1~ and \verb~V2~ (you can rename these, an
example is below).

You can perform computation on the whole \verb~data.table~ by leaving the
\verb~i~ argument empty. The number of females can be calculated like this:
\lstset{language=R,label= ,caption= ,numbers=none}
\begin{lstlisting}
d[, sum( Sex=="F" ) ]
\end{lstlisting}

\begin{verbatim}
[1] 2
\end{verbatim}

\subsection*{Use the third argument to group data}
\label{sec-2-4}

What if we want to calculate the mean age separately by sex? We can do
this in a single call using the \verb~by~ argument:
\lstset{language=R,label= ,caption= ,numbers=none}
\begin{lstlisting}
d[, mean(Age), by=Sex ]
\end{lstlisting}

\begin{verbatim}
   Sex   V1
1:   M 21.0
2:   F 25.5
\end{verbatim}

The result is another \verb~data.table~ with \verb~Sex~ as one column and the
result of the computation automatically named \verb~V1~. You can give a
better name as follows:
\lstset{language=R,label= ,caption= ,numbers=none}
\begin{lstlisting}
d2 <- d[, mean(Age), by=Sex ]
setnames( d2, "V1", "meanAge" )
d2
\end{lstlisting}

\begin{verbatim}
   Sex meanAge
1:   M    21.0
2:   F    25.5
\end{verbatim}

You can also group by multiple variables. Suppose you have this other \verb~data.frame~:

\begin{center}
\begin{tabular}{llrr}
Person & Sex & Age & Weight\\
\hline
John & M & 25 & 150\\
Jack & M & 22 & 170\\
Ann & F & 21 & 140\\
Sue & F & 25 & 145\\
Al & M & 22 & 180\\
Lucy & F & 21 & 160\\
\end{tabular}
\end{center}

You can calculate mean weight by age and sex as follows:
\lstset{language=R,label= ,caption= ,numbers=none}
\begin{lstlisting}
d[, mean(Weight), by=.(Age,Sex) ]
\end{lstlisting}

\begin{verbatim}
   Age Sex  V1
1:  25   M 150
2:  22   M 175
3:  21   F 150
4:  25   F 145
\end{verbatim}

Finally, note that the \verb~by~ argument can itself contain
computations. Suppose we want to split age in ``old'' and ``young'' using
a cutoff of 23. We can do it simply like this:
\lstset{language=R,label= ,caption= ,numbers=none}
\begin{lstlisting}
d[ , mean(Weight), by=Age<23 ]
\end{lstlisting}

\begin{verbatim}
     Age    V1
1: FALSE 147.5
2:  TRUE 162.5
\end{verbatim}

Note that the grouping variable \verb~Age~ retains its name, but its value
is the result of the computation in \verb~by~ rather then original age
value.
% Emacs 24.5.1 (Org mode 8.2.10)
\end{document}